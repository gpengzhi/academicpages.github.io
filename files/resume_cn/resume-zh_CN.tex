% !TEX TS-program = xelatex
% !TEX encoding = UTF-8 Unicode
% !Mode:: "TeX:UTF-8"

\documentclass{resume}
\usepackage{zh_CN-Adobefonts_external} % Simplified Chinese Support using external fonts (./fonts/zh_CN-Adobe/)
%\usepackage{zh_CN-Adobefonts_internal} % Simplified Chinese Support using system fonts
\usepackage{linespacing_fix} % disable extra space before next section
\usepackage{cite}

\begin{document}
\pagenumbering{gobble} % suppress displaying page number

\name{高鹏至}

% {E-mail}{mobilephone}{homepage}
% be careful of _ in emaill address
\contactInfo{(+1) 215-696-0238}{gpengzhi@gmail.com}{}{http://gpengzhi.github.io}
% {E-mail}{mobilephone}
% keep the last empty braces!
%\contactInfo{xxx@yuanbin.me}{(+86) 131-221-87xxx}{}
 
% \section{个人总结}
% 本人在校成绩优秀、乐观向上,工作负责、自我驱动力强、热爱尝试新事物,认同开放、连接、共享的Web在未来的不可替代性。在校期间长期从事可视分析(Web的2D/3D时空可视化)相关研究,对Web技术发展趋势及前端工程化解决方案有浓厚兴趣。\textbf{现任职于阿里巴巴集团。}

% \section{\faGraduationCap\ 教育背景}
\section{教育背景}
\datedsubsection{\textbf{伦斯勒理工学院},特洛伊,美国}
\ \textit{工学博士},电子工程,{2013.8 - 2017.12}
\begin{itemize}
	\item 导师:汪孟教授
	\item  博士论文: High-dimensional Data Analysis by Exploiting Low-dimensional Models with Applications in Synchrophasor Data Analysis in Power Systems
\end{itemize}
\datedsubsection{\textbf{宾夕法尼亚大学},费城,美国}
\ \textit{工学硕士},电子工程,{2011.8 - 2013.5}
\datedsubsection{\textbf{西安电子科技大学},西安,中国}
\ \textit{工学学士} (优秀毕业生),电子信息工程,{2007.8 - 2011.5}
% \ 2014年中国政府奖学金(\textit{http://www.csc.edu.cn/}),DID-ACTE项目交换生(\textit{http://did-acte.org/})

% \section{\faCogs\ IT 技能}
% \section{技术能力}
% increase linespacing [parsep=0.5ex]
% \begin{itemize}[parsep=0.2ex]
%   \item \textbf{编程语言}: JavaScript (ECMAScript, Node.js), HTML/CSS, Python, Go, SQL, C, Shell
%   \item \textbf{操作系统,数据库与工程构建}: Linux/macOS/MySQL/MongoDB/Git/webpack/Progressive Web App
%   \item \textbf{关键词}: React/Vue.js/D3.js(SVG)/three.js(canvas, WebGL)/chrome extension/Express
% \end{itemize}

% \end{itemize}


\section{研究方向}
我的研究方向包括信号处理,高维数据分析和机器学习。
%My research interests lie in the intersection of the fields of signal processing, high-dimensional statistics, and machine learning. % I am particularly interested in developing low-dimensional models and optimization methods with applications in image processing and power system monitoring.

\section{工作经历}
\datedsubsection{\textbf{槃腾科技 | Petuum},数据科学家}{2018.2-至今}
\begin{itemize}
\item 设计并开发用于 Petuum 人工智能开发平台的机器学习算法库 (基于 TensorFlow, Dynet 和 LightGBM)。
\item 设计并开发基于 PyTorch 的机器学习与文本生成工具箱 Texar-PyTorch (https://github.com/asyml/texar-pytorch,在 GitHub 上获得超过510 stars)。
\item 设计并开发用于文本处理的自然语言处理流水线工具 Forte (https://github.com/asyml/forte)。
\item 开发并维护基于 TensorFlow 的机器学习与文本生成工具箱 Texar-TensorFlow (https://github.com/asyml/texar,在GitHub上获得超过1790 stars)。
\end{itemize}

\datedsubsection{\textbf{微软亚洲研究院 |  Microsoft Research Asia},研究实习生}{2010.12-2011.5}
\begin{itemize}
\item 分析采集于Intel-Berkeley实验室的54个传感器的数据 (150MB),并探究其时域关联性。提出并开发了一种用于无线传感器网络数据采集的联合来源网络编码机制。
\end{itemize}


\section{技术能力}
% increase linespacing [parsep=0.5ex]
\begin{itemize}[parsep=0.2ex]
\item 熟练:MATLAB, Python, Dynet, PyTorch, TensorFlow
\item 有经验:Java, R, C/C++, C\#, AMPL
\end{itemize}


\section{获奖情况}
% increase linespacing [parsep=0.5ex]
\begin{itemize}[parsep=0.2ex]
\item \datedsubsection{IBM Watson Build Challenge 北美决赛入围者}{2017}
\item \datedsubsection{论文被选为 runner-up of the Best Paper in Electric Energy Systems Track of Hawaii International Conference on System Sciences}{2015}
\item \datedsubsection{Founders Award of Excellence (前1\%)}{2015}
\item \datedsubsection{论文被选为 one of the Best Conference Papers on Power System Analysis and Modeling of IEEE Power \& Energy Society General Meeting}{2014}
\item \datedsubsection{西安电子科技大学优秀毕业生 (前1\%)}{2011}
\item \datedsubsection{国家奖学金 (前1\%)}{2010}
\item \datedsubsection{西安电子科技大学一等奖学金 (前2\%)}{2008-2010}
\item \datedsubsection{西安电子科技大学学习标兵 (前1\%)}{2008}
\end{itemize}


\section{期刊论文}
% increase linespacing [parsep=0.5ex]
\begin{itemize}[parsep=0.2ex]
\item {{\bf P. Gao}, R. Wang, and M. Wang. ``Robust Matrix Completion by Exploiting Dynamic Low-dimensional Structures." \emph{submitted to IEEE Transactions on Signal Processing}, 2019.}

\item {{\bf P. Gao}, R. Wang, M. Wang, and J. H. Chow. ``Low-rank Matrix Recovery from Noisy, Quantized and Erroneous Measurements." \emph{IEEE Transactions on Signal Processing}, 2018, 66 (11): 2918-2932.}

\item {{\bf P. Gao}, M. Wang, J. H. Chow, M. Berger, and L. M. Seversky. ``Missing Data Recovery for High-dimensional Signals with Nonlinear Low-dimensional Structures." \emph{IEEE Transactions on Signal Processing}, 2017, 65 (20): 5421-5436.}

\item {{\bf P. Gao}, M. Wang, J. H. Chow, S. G. Ghiocel, B. Fardanesh, G. Stefopoulos, and M. P. Razanousky. ``Identification of Successive ``Unobservable'' Cyber Data Attacks in Power Systems Through Matrix Decomposition." \emph{IEEE Transactions on Signal Processing}, 2016, 64 (21): 5557-5570.}

\item {{\bf P. Gao}, M. Wang, S. G. Ghiocel, J. H. Chow, B. Fardanesh, and G. Stefopoulos. ``Missing Data Recovery by Exploiting Low-dimensionality in Power System Synchrophasor \\ Measurements." \emph{IEEE Transactions on Power Systems}, 2016, 31 (2): 1006-1013.}
\end{itemize}


\section{会议论文}
% increase linespacing [parsep=0.5ex]
\begin{itemize}[parsep=0.2ex]
\item {M. Wang, J. H. Chow, Y. Hao, S. Zhang, W. Li, R. Wang, {\bf P. Gao}, C. Lackner, E. Farantatos, and M. Patel. ``A Low-rank Framework of PMU Data Recovery and Event Identification.'' \emph{Proc. of the First IEEE International Conference on Smart Grid Synchronized Measurements and Analytics (SGSMA)}, College Station, Texas, May, 2019.}

\item {G. Mijolla, S. Konstantinouplos, {\bf P. Gao}, J. H. Chow, and M. Wang. ``An Evaluation of Low-Rank Matrix Completion Algorithms for Synchrophasor Missing Data Recovery." \emph{Proc. of the 20th Power Systems Computation Conference (PSCC)}, Dublin, Ireland, Jun. 2018.}

\item {{\bf P. Gao}, and M. Wang. ``Dynamic Matrix Recovery from Partially Observed and Erroneous Measurements." \emph{Proc. of the International Conference on Acoustics, Speech and Signal Processing (ICASSP)}, Calgary, Canada, Apr. 2018.}

\item {M. Wang, J. H. Chow, {\bf P. Gao}, Y. Hao, W. Li, and R. Wang. ``Recent Results of PMU Data Analytics by Exploiting Low-dimensional Structures." \emph{Proc. of the 10th Bulk Power Systems Dynamics and Control Symposium (IREP)}, Espinho, Portugal, Aug. 2017.}

\item {{\bf P. Gao}, R. Wang, and M. Wang. ``Quantized Low-rank Matrix Recovery with Erroneous Measurements: Application to Data Privacy in Power Grids." \emph{Proc. of Asilomar Conference on Signals, Systems, and Computers}, Pacific Grove, CA, Nov. 2016.}

\item {{\bf P. Gao}, M. Wang, and J. H. Chow. ``Matrix Completion with Columns in Union and Sums of Subspaces." \emph{Proc. of IEEE Global Conference on Signal and Information Processing (GlobalSIP)}, Orlando, FL, Dec. 2015.}

\item {M. Wang, J. H. Chow, {\bf P. Gao}, X. T. Jiang, Y. Xia, S. G. Ghiocel, B. Fardanesh, G. Stefopoulos, Y. Kokai, N. Saito, and M. P. Razanousky. ``A Low-Rank Matrix approach for the Analysis of Large Amounts of Synchrophasor Data." \emph{Proc. of Hawaii International Conference on System Sciences (\textcolor{red}{Runner-up of Best Paper in Electric Energy Systems Track})}, Kauai, Hawaii, Jan. 2015.}

\item {M. Wang, {\bf P. Gao}, S. G. Ghiocel, J. H. Chow, B. Fardanesh, G. Stefopoulos, and M. P. Razanousky. ``Identification of ``Unobservable'' Cyber Data Attacks on Power Grids." \emph{Proc. of IEEE SmartGridComm}, Venice, Italy, Nov. 2014.}

\item {{\bf P. Gao}, M. Wang, S. G. Ghiocel, and J. H. Chow. ``Modeless Reconstruction of Missing Synchrophasor Measurements." \emph{Proc. of IEEE Power \& Energy Society General Meeting (\textcolor{red}{selected in Best Conference Paper sessions})}, Washington, DC, Jul. 2014.}
\end{itemize}


\section{技术报告}
% increase linespacing [parsep=0.5ex]
\begin{itemize}[parsep=0.2ex]
\item {Zecong Hu, {\bf Pengzhi Gao}, Avinash Bukkittu, and Zhiting Hu. ``Introducing Texar-PyTorch: An ML Library integrating the best of TensorFlow into PyTorch." Octorber, 2019.}
\end{itemize}


\section{专利}
% increase linespacing [parsep=0.5ex]
\begin{itemize}[parsep=0.2ex]
\item {Meng Wang, {\bf Pengzhi Gao}, and Joe H. Chow. ``A low-rank-based missing PMU data recovery method." Application No.: 62/445305, Filed January 12, 2017.}
\end{itemize}

% \begin{onehalfspacing}
% \end{onehalfspacing}

% \datedsubsection{\textbf{DID-ACTE} 荷兰莱顿}{2015年3月 - 2015年6月}
% \role{本科毕业设计}{LIACS 交换生}
% 利用结巴分词对中国古文进行分词与词性标注,用已有领域知识训练形成 classifier 并对结果进行调优
% \begin{onehalfspacing}
% \begin{itemize}
%   \item 利用结巴分词对中国古文进行分词与词性标注
%   \item 利用已有领域知识训练形成 classifier, 并用分词结果进行测试反馈
%   \item 尝试不同规则,对 classifier 进行调优
% \end{itemize}
% \end{onehalfspacing}

\section{研究经历}
\datedsubsection{\textbf{伦斯勒理工学院}, 研究助理}{2013.8-2017.12}
\begin{itemize}
	\item 分析同步相量测量单元(PMU)数据(>200MB)并探究数据的时域与空间域的关联性(低维度特性)。
	\item 提出一种针对电网网络攻击的检测方法。用采集于纽约州中部电网的实际数据对我们的方法进行了测试。
	\item 开发了一种针对 PMU 数据的实时重构算法。在 OpenPDC 中用 C\#实现了相应的算法,并将计算时间减少了50\%。
	\item 提出了一种描述非线性数据的模型。基于这个模型,开发了一种凸优化数据恢复算法。用IEEE 39-bus New England Power System的标准测试数据集对我们的方法进行了测试。
	\item 针对量化及部分受损的数据,提出了一种数据恢复算法。用采集于纽约州中部电网的实际数据对我们的方法进行了测试。
\end{itemize}

\datedsubsection{\textbf{宾夕法尼亚大学}, 研究助理}{2012.5-2013.5}
\begin{itemize}
	\item 分析来自于 IEEG Portal的 EEG数据,并用于癫痫侦测任务。
	\item 提出一种针对于 EEG 数据的新字典,并将 EEG 数据恢复效果提升了20\%。
\end{itemize}

\section{专业活动}
\begin{itemize}
	\item IEEE 学生会员,2013 - 2017。IEEE 会员,2018 - 至今。
	\item  Center for Ultra-wide-area Resilient Electric Energy Transmission Networks (CURENT) 伦斯勒理工学院学生代表。
	\item 教学助理(伦斯勒理工学院):\\
	Modeling and Analysis of Uncertainty,2017年秋季学期, \\
	Distributed Systems and Sensor Networks,2017年秋季学期。
	\item 程序委员会成员:\\
	Conference on Uncertainty in Artificial Intelligence (UAI) 2018。
	\item 审稿人:\\
	IEEE Transactions on Smart Grid,\\
	IEEE Transactions on Automatic Control,\\
	IEEE/ACM Transactions on Networking,\\
	IEEE Signal Processing Letters,\\
	Annals of Mathematics and Artificial Intelligence,\\
	American Control Conference,\\
	IEEE International Conference on Communications, Control, and Computing Technologies
	for Smart Grids (SmartGridComm),\\
	International Symposium on Antennas and Propagation。
\end{itemize}


%% Reference
%\newpage
%\bibliographystyle{IEEETran}
%\bibliography{mycite}
\end{document}
